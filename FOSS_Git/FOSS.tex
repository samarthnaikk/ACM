\documentclass{article}
\usepackage{listings}
\usepackage{xcolor}

% Define colors for code highlighting
\definecolor{codegray}{gray}{0.9}
\definecolor{codegreen}{rgb}{0,0.6,0}
\definecolor{codeblue}{rgb}{0,0,1}

\lstdefinestyle{mystyle}{
    backgroundcolor=\color{codegray},
    commentstyle=\color{codegreen},
    keywordstyle=\color{codeblue},
    numberstyle=\tiny\color{gray},
    stringstyle=\color{codegreen},
    basicstyle=\ttfamily\footnotesize,
    breakatwhitespace=false,
    breaklines=true,
    captionpos=b,
    keepspaces=true,
    numbers=left,
    numbersep=5pt,
    showspaces=false,
    showstringspaces=false,
    showtabs=false,
    tabsize=2
}

\lstset{style=mystyle}

\begin{document}

\title{Advanced Git Commands}
\author{}
\date{}
\maketitle

\section{Introduction}
Git provides many powerful commands beyond the commonly used ones like \texttt{commit}, \texttt{push}, and \texttt{merge}. This document covers nine lesser-known but useful Git commands, including their syntax, use cases, and examples.

\section{git worktree}
\textbf{Description:} Allows multiple working directories from a single repository.

\textbf{Syntax:}
\begin{lstlisting}
git worktree add <path> <branch>
\end{lstlisting}

\textbf{Use Cases:}
\begin{itemize}
    \item Work on multiple branches simultaneously.
    \item Maintain separate environments for testing.
\end{itemize}

\textbf{Example:}
\begin{lstlisting}
git worktree add feature-work feature-branch
cd feature-work
\end{lstlisting}

\section{git bisect}
\textbf{Description:} Finds the commit that introduced a bug via binary search.

\textbf{Syntax:}
\begin{lstlisting}
git bisect start
git bisect good <commit>
git bisect bad <commit>
git bisect reset
\end{lstlisting}

\section{git rerere}
\textbf{Description:} Remembers conflict resolutions to reuse them later.

\textbf{Syntax:}
\begin{lstlisting}
git config --global rerere.enabled true
git merge <branch>
\end{lstlisting}

\section{git reflog}
\textbf{Description:} Tracks all changes to the branch references.

\textbf{Syntax:}
\begin{lstlisting}
git reflog
git checkout HEAD@{n}
\end{lstlisting}

\section{git replace}
\textbf{Description:} Temporarily substitutes one commit for another.

\textbf{Syntax:}
\begin{lstlisting}
git replace <old-commit> <new-commit>
\end{lstlisting}

\section{git cherry-pick}
\textbf{Description:} Applies a specific commit from another branch.

\textbf{Syntax:}
\begin{lstlisting}
git cherry-pick <commit>
\end{lstlisting}

\section{git submodule}
\textbf{Description:} Manages repositories inside repositories.

\textbf{Syntax:}
\begin{lstlisting}
git submodule add <repo-url> <path>
\end{lstlisting}

\textbf{Use Cases:}
\begin{itemize}
    \item Include external libraries inside a repository.
    \item Track dependencies separately from the main project.
\end{itemize}

\section{git blame}
\textbf{Description:} Shows who made changes to each line in a file.

\textbf{Syntax:}
\begin{lstlisting}
git blame <file>
\end{lstlisting}

\textbf{Use Cases:}
\begin{itemize}
    \item Identify who introduced a particular change.
    \item Debugging code history.
\end{itemize}

\section{git bundle}
\textbf{Description:} Creates a single file containing Git repository data.

\textbf{Syntax:}
\begin{lstlisting}
git bundle create <file> <branch>
\end{lstlisting}

\textbf{Use Cases:}
\begin{itemize}
    \item Share repositories without direct access.
    \item Archive a repository snapshot.
\end{itemize}

\section{Conclusion}
These advanced Git commands can enhance productivity, improve debugging, and streamline workflows. Mastering them can help developers manage complex Git projects more efficiently.

\end{document}
